
\documentclass[11pt]{article}
\usepackage[utf8]{inputenc}
\usepackage{geometry}
\geometry{ a4paper, total = {170mm, 240mm}, left = 20mm, top = 30mm,}
\usepackage{times}
\usepackage[unicode]{hyperref}
\usepackage[czech]{babel}
\usepackage{array}
\usepackage{multirow}
\usepackage[linesnumbered,ruled,noline, longend]{algorithm2e}
\usepackage{amsmath}
\usepackage{amsthm}
\usepackage{amsfonts}
\usepackage{amssymb}
\usepackage{graphicx}
\usepackage{pdflscape}
\usepackage{pict2e}

\begin{document}
\begin{titlepage}
\begin{center}
    \textsc{\Huge{Vysoké učení technické v~Brně} \vspace{0.4em} \\
    \huge{Fakulta informačních technologií}}\\
    \vspace{\stretch{0.382}}
    {\LARGE{Typografie a publikování\,--\,3.\,projekt}\vspace{0.3em}\\
    \Huge{Tabulky a obrázky}}    
    \vspace{\stretch{0.618}}
\end{center}
{\Large \today \hfill David Kocman}
\end{titlepage}

\section{Úvodní strana}
    Název práce umístěte do zlatého řezu a nezapomeňte uvést dnešní datum a vaše jméno a příjmení.
    
\section{Tabulky}
    Pro sázení tabulek můžeme použít bud’ prostředí \texttt{ tabbing } nebo prostředí \texttt{ tabular}.
    
\subsection{Prostředí \texttt{tabbing}}
    Při použití \texttt{ tabbing } vypadá tabulka následovně:
    
   \begin{tabbing}
        Vodní melouny\quad \=\textbf{Cena} \quad \= \textbf{Množství}  \quad  \kill
        \textbf{Ovoce} \> \textbf{Cena} \> \textbf{Množství}  \\
        Jablka \> 25,90 \> 3~kg \\
        Hrušky \> 27,40 \> 2,5~kg \\
        Vodní melouny \> 35,- \> 1 kus \\
\end{tabbing}

\noindent{Toto prostředí se dá také použít pro sázení algoritmů, ovšem vhodnější je použít 
prostředí \texttt{algorithm} nebo \texttt{algorithm2e} (viz sekce 3).}

\subsection{Prostředí \texttt{tabular}}
    Další možností, jak vytvořit tabulku, je použít prostředí tabular. Tabulky pak 
budou vypadat takto\footnote{Kdyby byl problém s~\texttt{ cline, } zkuste se podívat třeba sem: \href{http://www.abclinuxu.cz/tex/poradna/show/325037}{http://www.abclinuxu.cz/tex/poradna/show/325037 }.}: \bigskip

\begin{table}[h]
\catcode`\-=12
\begin{center}
    \begin{tabular}{|c|c|c|} \hline
         &\multicolumn{2}{|c|}{\textbf{Cena}}  \\ \cline{2-3}
        \textbf{Měna} & \textbf{nákup} & \textbf{prodej}
        \\
        \hline
        EUR & 25,227 & 26,943 \\ 
        GBP & 29,368 & 31,492 \\
        USD & 21,260 & 22,661 \\ \hline
    \end{tabular}
    \caption{Tabulka kurzů k~dnešnímu dni}
    \label{tabKurzu}
\end{center}
\end{table}
\begin{table}[h] 
\catcode`\-=12
\begin{center}
    \begin{tabular}{|c|c|} \hline
         $A$ & $\neg A$ \\ \hline
         P & N\\ \hline
         O~& O\\ \hline
         X & X\\ \hline
         N & P\\ \hline
    \end{tabular}
    \begin{tabular}{|c|c|c|c|c|c|} \hline
        \multicolumn{2}{|c|}{\multirow{2}{*}{$A \land B$}} & \multicolumn{4}{|c|}{$B$} \\ \cline{3-6}
        \multicolumn{2}{|c|}{} & \textbf{P} & \textbf{O} & \textbf{X} & \textbf{N} \\ \cline{1-6}
        \multirow{4}{*}{$A$}& \textbf{P} & P & O~& X & N \\ \cline{2-6}
                            & \textbf{O} & O~& O~& N & N\\ \cline{2-6}
                            & \textbf{X} & X & N & X & N\\\cline{2-6}
                            & \textbf{N} & N & N & N & N\\ \hline
    \end{tabular}
    \begin{tabular}{|c|c|c|c|c|c|} \hline
        \multicolumn{2}{|c|}{\multirow{2}{*}{$A \lor B$}} & \multicolumn{4}{|c|}{$B$} \\ \cline{3-6}
        \multicolumn{2}{|c|}{} & \textbf{P} & \textbf{O} & \textbf{X} & \textbf{N} \\ \cline{1-6}
        \multirow{4}{*}{$A$}& \textbf{P} & P & P & P & P \\ \cline{2-6}
                            & \textbf{O} & P & O~& P & O\\ \cline{2-6}
                            & \textbf{X} & P & P & X & X\\\cline{2-6}
                            & \textbf{N} & P & O~& X & N\\ \hline
    \end{tabular}
    \begin{tabular}{|c|c|c|c|c|c|} \hline
        \multicolumn{2}{|c|}{\multirow{2}{*}{$A \to B$}} & \multicolumn{4}{|c|}{$B$} \\ \cline{3-6}
        \multicolumn{2}{|c|}{} & \textbf{P} & \textbf{O} & \textbf{X} & \textbf{N} \\ \cline{1-6}
        \multirow{4}{*}{$A$}& \textbf{P} & P & O~& X & N \\ \cline{2-6}
                            & \textbf{O} & P & O~& P & O\\ \cline{2-6}
                            & \textbf{X} & P & P & X & X\\\cline{2-6}
                            & \textbf{N} & P & P & P & P\\ \hline
    \end{tabular}
    \caption{Protože Kleeneho trojhodnotová logika už je \uv{zastaralá}, uvádíme si zde příklad čtyřhodnotové logiky}
    \label{tabLogiky}
\end{center}
\end{table}
\pagebreak

\section{Algoritmy}

Pokud budeme chtít vysázet algoritmus, můžeme použít prostředí \texttt{ algorithm}\footnote{Pro nápovědu, jak zacházet s~prostředím \texttt{algorithm}, můžeme zkusit tuhle stránku:\\ \href{http://ftp.cstug.cz/pub/tex/CTAN/macros/latex/contrib/algorithms/algorithms.pdf}{http://ftp.cstug.cz/pub/tex/CTAN/macros/latex/contrib/algorithms/algorithms.pdf}
.} nebo \texttt{algorithm2e}\footnote{
Pro \texttt{ algorithm2e } zase tuhle: \href{http://ftp.cstug.cz/pub/tex/CTAN/macros/latex/contrib/algorithm2e/doc/algorithm2e.pdf}{http://ftp.cstug.cz/pub/tex/CTAN/macros/latex/contrib/algorithm2e/doc/algorithm2e.pdf}}. Příklad použití prostředí \texttt{ algorithm2e } viz Algoritmus \ref{Algoritmus}. \bigskip

\begin{algorithm}[H] 
    \SetNlSty{}{}{:}
    
    \KwIn{$(X_{t-1}, u_t, z_t)$}
    \KwOut{$X_t$}
    \Indp
    \SetNlSkip{-1em}
    \BlankLine
    $\overline{X_t} = X_t = 0$\\
    \For{$k=1$ \emph{to} $M$}{
        $x_k^{[k]} = sample\_motion\_model(u_t, x_{t-1}^{[k]})$\\
        $\omega_t^{[t]} = measurement\_model(t_t, x_t^{[k]}, m_{t-1})$\\
        $m_t^{[k]} = updated\_occupancy\_grid(z_t, x_t^{[k]}, m_{t-1}^{[k]})$\\
        $\overline{X_t} = \overline{X_t} + \langle x_x^{[m]}, \omega_t^{[m]} \rangle$\\
    }
    \For{$k=1$ \emph{to} $M$}{
        draw $i$ with propability $\approx \omega_t^{[i]}$\\
        add $\langle x_x^{[k]},m_t^{[x]} \rangle$ to $X_t$        
    }
    \Return{$X_t$}
    \caption{\textsc{FastSLAM}}
    \label{Algoritmus}
\end{algorithm}

\section{Obrázky}
Do našich článků můžeme samozřejmě vkládat obrázky. Pokud je obrázkem fotografie,
můžeme klidně použít bitmapový soubor. Pokud by to ale mělo být nějaké schéma nebo
něco podobného, je dobrým zvykem takovýto obrázek vytvořit vektorově.

\begin{figure}[h]
    \centering
    \scalebox{0.43}{\includegraphics{etiopan.eps}
    \reflectbox{\includegraphics{etiopan.eps}}}
    \caption{Malý etiopiánek a jeho bratříček}
    \label{Obrazek 1}
\end{figure}
\newpage
Rozdíl mezi vektorovým \dots

\begin{figure}[h]
    \centering
    \scalebox{0.5}{\includegraphics{oniisan.eps}}
    \caption{Vektorový obrázek}
    \label{Obrazek 2}
\end{figure}

\dots a bitmapovým obrázkem

\begin{figure}[h]
    \centering
    \scalebox{0.8}{\includegraphics{oniisan2.eps}}
    \caption{Bitmapový obrázek}
    \label{Obrazek 3}
\end{figure}

se projeví až při zvětšení

Odkazy (nejen ty) na obrázky \ref{Obrazek 1}, \ref{Obrazek 2} a \ref{Obrazek 3}, na  
tabulky \ref{tabKurzu} a \ref{tabLogiky} a také na algoritmus \ref{Algoritmus} jsou udělány pomocí 
křížových odkazů. Pak je ovšem potřeba zdrojový soubor přeložit dvakrát.

Vektorové obrázky lze vytvořit i přímo v~\LaTeX u, například pomocí prostředí \texttt{ picture}.
\newpage
\begin{landscape}
\begin{figure}
\begin{center}
        \begin{minipage}[c]{\textwidth}
        \vspace{175pt}
        \begin{picture}(20,10)
            \setlength{\unitlength}{1cm}
            \put(0,0){\line(1,0){20}}
            \put(0,0){\line(0,1){10}}
            \put(0,10){\line(1,0){20}}
            \put(20,0){\line(0,1){10}}
            \linethickness{5pt}
            \put(1,1){\line(1,0){18}}
            \linethickness{1pt}
            \put(2,1){\line(0,1){2.5}}
            \put(2,3.5){\line(1,0){11.5}}
            \put(13.5,1){\line(0,1){2.5}}
            \put(15,1){\line(0,1){1.75}}
            \put(13.5,2.75){\line(1,0){1.5}}
            \put(8,3.5){\line(0,1){2.25}}
            \put(8.75,3.5){\line(0,1){1.5}}
            \put(8,5.75){\line(1,0){10}}
            \put(8.75,5){\line(1,0){8.5}}
            \put(18,1){\line(0,1){4.75}}
            \put(17.25,1){\line(0,1){4}}
            \put(13.5,3.1){\line(1,0){3.75}}
            \put(9.5,1){\line(0,1){1.5}}
            \put(11.5,1){\line(0,1){1.5}}
            \put(9.5,2.5){\line(1,0){2}}
            \put(7,1){\line(0,1){1.5}}
            \put(7.75,1){\line(0,1){1.5}}
            \put(7,2.5){\line(1,0){0.75}}
            \multiput(13.5,1.25)(0,0.25){6}{\line(1,0){1.5}}
            \put(11.5,5.75){\line(0,1){0.75}}
            \put(11.5,6.5){\line(1,0){3}}
            \put(14.5,5.75){\line(0,1){0.75}}
            \put(3.2,7.5){\circle{2}}
        \end{picture}
        \caption{Vektorový obrázek mého domu vytvořený v~prostředí \texttt{ picture}}
        \label{fig:my_label}
        \end{minipage}
    \end{center}
    \end{figure}
\end{landscape}
\end{document}