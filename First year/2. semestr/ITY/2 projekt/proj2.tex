\documentclass[11pt, twocolumn]{article}
\usepackage[utf8]{inputenc}
\usepackage[czech]{babel}
\usepackage[IL2]{fontenc}
\usepackage{geometry}
\geometry{ a4paper, total = {180mm, 250mm}, left = 15mm, top = 25mm,}
\usepackage{amsmath}
\usepackage{amsthm}
\usepackage{amsfonts}
\usepackage{amssymb}
\usepackage{times}
\newtheorem{definition}{Definice}
\newtheorem{theorem}{Věta}
\usepackage[unicode]{hyperref}

\begin{document}
\begin{titlepage}
\begin{center}
    {\Huge \textsc{Fakulta informačních technologií\vspace{0.4em} \\ Vysoké učení technické v~Brně}}\\
    \vspace{\stretch{0.382}}
    {\huge Typografie a publikování\,--\,2.\,projekt\vspace{0.3em}\\
    Sazba dokumentů a matematických výrazů}
    \vspace{\stretch{0.618}}
\end{center}
    {\LARGE 2021 \hfill David Kocman (xkocma08)}

\end{titlepage}

\section*{Úvod}
V~této úloze si vyzkoušíme sazbu titulní strany, matematic\-kých vzorců, prostředí a dalších textových struktur obvyk\-lých pro technicky zaměřené texty (například rovnice (\ref{equation 1})
nebo Definice 1 na straně \pageref{Definice 1}). Rovněž si vyzkoušíme pou\-žívání odkazů \verb|\ref| a \verb|\pageref|.

Na titulní straně je využito sázení nadpisu podle op\-tického středu s~využitím zlatého řezu. Tento postup byl
probírán na přednášce. Dále je použito odřádkování se
zadanou relativní velikostí 0.4 em a 0.3 em.

V~případě, že budete potřebovat vyjádřit matematickou
konstrukci nebo symbol a nebude se Vám dařit jej nalézt
v~samotném \LaTeX u, doporučuji prostudovat možnosti ba\-líku maker \AmS-\LaTeX.

\section{Matematický text}

Nejprve se podíváme na sázení matematických symbolů
a výrazů v~plynulém textu včetně sazby definic a vět s~vy\-užitím balíku \verb|amsthm|. Rovněž použijeme poznámku pod
čarou s~použitím příkazu \verb|\footnote|. Někdy je vhodné
použít konstrukci \verb|\mbox{}|, která říká, že text nemá být
zalomen.

\begin{definition}
\label{Definice 1} Rozšířený zásobníkový automat (RZA) je de\-finován jako sedmice tvaru $A = (Q,\Sigma,\Gamma,\delta,q_0,Z_0,F)$,
kde:
\end{definition}
\begin{itemize}
    \item \emph{Q je konečná množina} vnitřních (řídicích) stavů,
    
    \item \emph{$\Sigma$ je konečná} vstupní abeceda, 
    
    \item \emph{$\Gamma$ je konečná} zásobníková abeceda,
    
    \item \emph{$\delta$ je} přechodová funkce $Q\times(\Sigma\cup\{\epsilon\})\times\Gamma^* \rightarrow 2^{Q\times\Gamma^*}$,
    
    \item $q_0 \in Q$ je počáteční stav,$Z_0 \in \Gamma$ je startovací symbol
    zásobníku a $F \subseteq Q$ je \emph{množina} koncových stavů.

\end{itemize}

Nechť $P = (Q, \Sigma, \Gamma, \delta, q_0, Z_0, F)$ je rozšířený zásob\-níkový automat. \emph{Konfigurací} nazveme trojici $(q, w, \alpha) \in
Q \times \Sigma^* \times \Gamma^*$
, kde $q$ je aktuální stav vnitřního řízení,
\begin{math}w\end{math} je dosud nezpracovaná část vstupního řetězce a $\alpha\,=\,Z_{i_1} Z_{i_2} \ldots Z_{i_k}$ je obsah zásobníku\footnote{$Z_{i_1}$ je vrchol zásobníku}.

\subsection{Podsekce obsahující větu a odkaz}

\begin{definition}
 \label{Definice 2} \emph{Řetězec $w$ nad abecedou $\Sigma$ je přijat RZA} $A$~jestliže $(q_0, w, Z_0) \underset{A}{\overset{*}{\vdash}} (q_F , \epsilon, \gamma)$ pro nějaké $\gamma \in \Gamma^*$ a $q_F \in F$. Množinu $L(A) = \{w~|~w \mbox{ je přijat RZA A}\}$ $\subseteq \Sigma^*$ nazýváme \emph{jazyk přijímaný RZA $A$.}
\end{definition}
Nyní si vyzkoušíme sazbu vět a důkazů opět s~použitím
balíku \verb|amsthm|.

\begin{theorem}
Třída jazyků, které jsou přijímány ZA, odpovídá
\emph{bezkontextovým jazykům.}
\end{theorem}
\begin{proof}
V~důkaze vyjdeme z~Definice \ref{Definice 1} a \ref{Definice 2}.
\end{proof}

\section{Rovnice a odkazy}

Složitější matematické formulace sázíme mimo plynulý
text. Lze umístit několik výrazů na jeden řádek, ale pak je
třeba tyto vhodně oddělit, například příkazem \verb|\quad|.

$$\sqrt[i]{x_i^3}\quad \text{ kde } x_i \text{ je } i\text{-té sudé číslo splňující}\quad x_i^{x_i^{i^2} + 2}\leq y_i^{x_i^4}$$

V~rovnici (\ref{equation 1}) jsou využity tři typy závorek s~různou
explicitně definovanou velikostí.

\begin{equation}\label{equation 1}
    x = \bigg[ \Big\{ \big[a + b\big] \ast c \Big\} ^d \oplus 2 \bigg] ^{3/2}
\end{equation}


    $$y = \lim\limits _{x \rightarrow \infty} \frac{\frac{1}{\log _{10} x}}{\sin ^2 x + \cos ^2 x}$$

V~této větě vidíme, jak vypadá implicitní vysázení li\-mity $\lim _{n \rightarrow \infty} f(n)$ v~normálním odstavci textu. Podobně
je to i s~dalšími symboly jako $\prod_{i = 1} ^n 2^i$ či $\bigcap _{A\in B} A$. V~pří\-padě vzorců $\lim\limits _{n \rightarrow \infty} f(n)$ a $\prod\limits_{i = 1} ^n 2^i$ jsme si vynutili méně úspornou sazbu příkazem \verb|\limits|. 

\begin{equation}
    \int _b ^a g(x)\, \mathrm{d}x \; =\; - \int\limits _a ^b f(x)\, \mathrm{d}x
\end{equation}

\section{Matice}
Pro sázení matic se velmi často používá prostředí \verb|array|
a závorky (\verb|\left|,\verb|\right|)

$$\left(
\begin{array}{ccc}
    a - b & \widehat{\xi + \omega} & \pi \\
    \Vec{\mathbf{a}} & \overleftrightarrow{AC} & \hat{\beta}
\end{array} 
\right)
= 1 \Longleftrightarrow \mathcal{Q} = \mathbb{R}$$
$$\mathbf{A} \left \|
\begin{array}{cccc}
    a_{11} & a_{12} & \cdots & a_{1n} \\
    a_{21} & a_{22} & \cdots & a_{2n} \\
    \vdots &\vdots &\ddots&\vdots\\
    a_{m1} & a_{m2} & \cdots & a_{mn} \\
\end{array}
\right \|
 = \left|
\begin{array}{cc}
     t & u\\
     v~& w
\end{array}
\right| = tw - uv$$

Prostředí \verb|array| lze úspěšně využít i jinde.
$$\left(
\begin{array}{l}
n\\
k
\end{array}
\right) = \left\{
\begin{array}{c l}
    0& \text{pro } k~\le 0 \text{ nebo } k~\ge n\\
    \frac{n!}{k!(n-k)!}& \text{pro } 0 \leq k~\leq n
\end{array}
\right .
$$
\end{document}
