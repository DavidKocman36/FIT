\documentclass{beamer}
\usepackage{graphicx}
\usepackage[czech]{babel}
\usepackage[utf8]{inputenc}


\title{Typografie a publikování}
\subtitle{5. projekt -- Prezentace}
\author{David Kocman, xkocma08}
\institute{FIT VUT}
\date{2021}


\begin{document}

\frame{\titlepage}

\begin{frame}
\frametitle{Pole}
    \bigskip
    \begin{itemize}
        \item Pole je jedním z~nejstarších a nejpoužívanějších datových struktur
        \item Vzniklo proto, aby se mohli reprezentovat vektory
        \item Je to konečné seskupení prvků jednoho datového typu
        \item Prvky pole mohou být také pole
        \item Rozlišují se na dynamická a statická pole
    \end{itemize}
\end{frame}
\begin{frame}{Práce s~poli}
    \begin{itemize}
        \item Prvky v~poli mají indexy, počínaje vždy 0
        \item Můžeme s~polem provádět následující operace:
        \begin{itemize}
            \item Vypiš pole
            \item Vlož prvek
            \item Smaž prvek
            \item Vyhledej prvek
            \item Změň prvek
            \item Seřaď prvky
        \end{itemize}
    \end{itemize}
\end{frame}
\begin{frame}{Práce s~poli\,--\,příklad}
    \begin{itemize}
        \item Příklad vypsání pole:
    \end{itemize}
    \bigskip
    \texttt{pole [] = \{1,2,3\};}\\
    \texttt{for(i < sizeof(pole))}\\
    \{\\
        \qquad\texttt{printf("\%d ", pole[i]);}\\
    \}
\end{frame}

\begin{frame}{Práce s~poli\,--\,příklad}
    \begin{itemize}
        \item Příklad vyhledání prvku:
    \end{itemize}
    \bigskip
    \texttt{pole [] = \{1,2,3\};}\\
    \texttt{found = false;}\\
    \texttt{var = 2;}\\
    \texttt{for(i < sizeof(pole))}\\
    \{\\
        \qquad\texttt{if(pole[i] == var)}\\
        \qquad\{\\
        \qquad\qquad\texttt{found = true;}\\
        \qquad\qquad\texttt{break;}\\
        \qquad\}\\
    \}
\end{frame}

\begin{frame}{Statické pole}
    \begin{itemize}
        \item Statické pole má fixní počet prvků a není možné jej měnit při běhu
        \item Inicializuje a deklaruje se před během programu
    \end{itemize}
    \bigskip
    \texttt{int pole[10]=\{1,2,3\};}
\end{frame}
\begin{frame}{Dynamické pole}
    \begin{itemize}
        \item Dynamické pole nemusí mít pevný počet prvků
        \item Naplňuje se při běhu programu, např funkcí \texttt{malloc()}
        \item Deklarované však může být před během aplikace
        \item Po dokončení běhu programu je vhodné vždy alokované zdroje uvolnit funkcí \texttt{free()}
    \end{itemize}
    \bigskip
    \texttt{int a [];}\\
    \texttt{for(i < sizeof(pole)}\\
    \{\\
        \qquad\texttt{a[i] = malloc(sizeof(int));}\\
        \qquad\texttt{a[i] = i;}\\
    \}\\
\end{frame}

\begin{frame}{Pole}
    \begin{figure}[h]
        \centering
        \scalebox{0.75}{\includegraphics{array.eps}}
        \caption{Příklad pole, angl. Array}
    \end{figure}
\end{frame}

\begin{frame}{Použití polí}
    \begin{itemize}
        \item V~jazyce C slouží pole charů jako string
        \item Seznam
        \item Ukládání více argumentů stejného typu pro příští užití např. v~loopu
        \item Reprezentace vektorů
    \end{itemize}
\end{frame}

\begin{frame}{Zdroje}
\begin{itemize}
    \item \href{https://www.tutorialsteacher.com/csharp/array-csharp}{https://www.tutorialsteacher.com/csharp/array-csharp}
    \item \href{https://www.itnetwork.cz/navrh/algoritmy/algoritmy-datove-struktury/datove-struktury-pole-a-list}{https://www.itnetwork.cz/navrh/algoritmy/algoritmy-datove-struktury/datove-struktury-pole-a-list}
\end{itemize}
\end{frame}
\end{document}
